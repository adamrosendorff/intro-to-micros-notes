\chapter{Clock Distribution}
In order for a block of circuitry to function inside the microcontroller it needs to be clocked. Clocking circuitry provides it with well defined timing which allows the circuitry to take data from the bus or place data onto the bus exactly in sync with all other circuitry in the microcontroller. This essentially enables the circuitry for use. However, as soon as circuitry inside the micro is clocked/enabled, it draws power. For that reason, each internal peripheral can selectively be enabled or disabled by providing or removing (better known as \emph{gating}) the clock to that peripheral. Most importantly the clock is default \emph{OFF} for all peripherals in order to make the default power consumption as low as possible. The exact amount of power consumed by each peripheral is different depending on which peripheral it is. Furthermore, power consumption is approximately linear with clock speed. At maximum clock speed the power consumption is roughly half a milliamp per peripheral. 

\section{Reset and Clock Control}
The RCC is a peripheral. As the name implies, one of its key functions in the management of the clocking system of the microcontroller. This involves both generating or altering the clock frequency and selectively gating or allowing clock to the other peripherals of the micro. 

The peripherals are divided up into a bus structure as shown earlier and as such the structure of the clock distribution is also based on a bus structure. The RCC has a register for each bus. The register controls the state of the clock of the devices connected to that bus. These registers include the RCC\_AHBENR, RCC\_APB1ENR and RCC\_APB2ENR. 

\subsection{Clock Source}
As well as managing the gating of clocks for peripherals the RCC also selects the oscillator which should be the source of the system clock (sysclock). The default source is an internal 8 MHz RC oscillator. Optionally, the external crystal quartz oscillator can be selected as clock source. The advantage of using an internal oscillator is that it does not require an extra component to be connected to the micro. The disadvantage is the tolerance: it is around 1\% at room temperature, but can be more than 4\% at more extreme temperatures. The tolerance of an external crystal quartz oscillator is typically much better than 0.01\%. Hence, for applications which are not timing sensitive the internal oscillator can be used. For timing sensitive applications the external oscillator should be used. 

\section{CPU Instruction Cycles}
As well as driving the peripherals the system clock drives the CPU. The instructions which we write each take a certain number of CPU cycles to execute. Instructions typically take one cycle to execute, but this is not true for all instructions. The exact number of cycles which can instruction takes is detailed in Section 3.3 of the ARM Cortex-M0 Technical Reference Manual. By knowing how many cycles each instruction takes to execute we can know how many cycles a specific block of code takes to execute. By knowing how long each cycles takes in real time (this is the inverse of frequency) we know the real time which a block of code takes to execute. 
