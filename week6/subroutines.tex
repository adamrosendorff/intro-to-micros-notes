\chapter{Subroutines}
It would be very useful to have the ability to branch to a label, execute a block of code and then return back to where the branch was taken from. A block of code which is branched to and returned from in this way is called a subroutine.
Subroutines are a very useful concept as they allow us to write a single block of code and then re-use it multiple times. 
If we did not have subroutines we would have to duplicate code whenever we wanted to make use of the functionality provided by the code.
This causes unnecessary use (wasting) of flash memory.
Also, it makes the job of maintaning the code very difficult because if you want to adjust something in that block of code then you would have to make the adjustments in multiple places in your source file.
By having a subroutines the code only occupies space in memory once and alterations to it only have to happen in one place.

In order to implement this subroutine concept the CPU needs the ability to store the return address somewhere when a subroutine is branched to. 
Subroutines are so useful that an entire CPU register is dedicated to the purpose of storing return addresses for subroutines. That is R14, otherwise known as the Link Register (LR).
Subroutines work by storing the address of the next instruction to be executed in the LR and then branching to the label of the subroutine.
When you want to "return" from the subroutine you simply move the data in the LR into the PC which causes the CPU to go back to pointing to the next instruction to be executed after the branch.

In order to get the branch instruction to store the address of the next instruction in the LR, the following instruction format is used.
BL label

In order to load the contents of an arbitrary register into the PC, the Branch Indirect (BX) instruction is used.
The general format of this instruction is:
\begin{lstlisting}[fontadjust=true,frame=trBL]
BX Rn  @ where Rn is some register
\end{lstlisting}

To return from a subroutine we want to move the contents of the LR into the PC. Hence the specific format of the intruction to use is:
\begin{lstlisting}[fontadjust=true,frame=trBL]
BX LR
BX R14
\end{lstlisting}
