\section{Defining a Microcontroller}

\subsection{What is a microcontroller?}
The microcontroller can be understood by comparing it to something you are already very familiar with: the computer. Both a microcontroller and a computer can be modeled as a black box which takes in data and instructions, performs processing, and provides output.

\subsection{A basic model of the STM32F051}

\subsection{The ARM Cortex-M0}
The microcontroller which we will be using is the STM32F051C6. At the core of this micro is it's CPU, which is called the Cortex-M0 and is designed by Advanced RISC Machines (ARM).a

\subsection{A Short History of ARM}
Acorn

\subsection{32-bit Processor?}
It's been said that the ARM Cortex-M0 is a 32-bit processor. For comparison, the procesor which we used in this course previously (MC9S08GT16A) was an 8-bit processor. Your personal computer probably has a 64-bit CPU. 16-bit CPUs are also quite common. So what exactly does it mean when we say that the processor is 32-bits? Essentially, the number of bits which a processor is said to be referes to the size of the data bus. In other words: the amount of data which the processor is able to move around internally or perform arithmetic and logic operations on. Hence, with a 32-bit processor, we can move 32 bits of data from one spot in memory to another in just once instruction. If you had a 8-bit processor, it would cost 4 instructions to move 32 bits of data around.  
