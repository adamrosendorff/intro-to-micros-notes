\section{A History of Processing}
In 1971 the dawn of a new era began. Intel had just announced that they had developed ``a programmable computer on a chip.'' The chip, known as the 4004 was the first general purpose microprocessor on one silicon chip, and contained around 2300 transistors. In order to produce this chip, the layout of the transistors was hand-drawn using coloured pencils at 500 times scale. The CPU had the ability to transfer 4 bits per clock cycle, had a 12-bit address space and an 8-bit instruction. In total, it has 16 instructions which it could execute. It was used in a calculator which had an optional square root function. The calculator run around 100K instructions per second, had 1 KB of ROM and 80 bytes of RAM. 

The next year, in 1972 Intel released the 8008, an 8-bit microprocessor. The 8008 had a 14 bit address space. The first use for the 8008 was a programmable scientific calculator. In 1973 the 8008 was used as the CPU for a French desktop computer. Some consider this to be the first desktop computer. The 8008 evolved into the 8080. 

While desktop computers do take a large share of the microprocessor market, many times more microprocessors go into microcontrollers for the purpose of embedded control applications. In 1996, 25 years after the advent of the first microprocessor, 70\% of all semiconductors are used for microcontroller-based circuits. 

A processor with IO an memory is a microcontroller. In 1995, 50\% of microcontrollers manufacturered were 4-bit. 
